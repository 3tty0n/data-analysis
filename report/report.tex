\documentclass[a4paper,xelatex,ja=standard,jafont=hiragino-pron]{bxjsarticle}

\usepackage{xltxtra}
\XeTeXlinebreaklocale "ja"

\usepackage{amsmath, amssymb}
\topmargin = 0mm
\oddsidemargin = 5mm
\textwidth = 152mm
\textheight = 240mm
\renewcommand{\thesection}{問\arabic{section}}
\renewcommand{\thesubsection}{(\arabic{subsection})}
\renewcommand{\thesubsubsection}{(\arabic{subsection}.\arabic{subsubsection})}

\title{データ解析レポート課題・第一}
\author{14\_01043 伊澤 侑祐}
\date{}

\begin{document}
\maketitle
\section{計算問題}
\subsection{$E(a)$を最小にするパラメータ$a^* = \{(a^*)_k$\}を求めよ}
まず、$k$を固定する。二乗誤差

\begin{eqnarray}
  E(a) &=& \frac{1}{n} \sum_{i = 1}^n (Y_i - f(x_i, a))^2 \nonumber \\
       &=& \frac{1}{n} \sum_{i = 1}^n (Y_i - \sum_{k = 1}^K a_k e_k(x))^2
\end{eqnarray}

を$a_k$について偏微分し、$\frac{\partial E}{\partial a_k} = 0$を解く。

\begin{eqnarray}
  &&\frac{\partial E}{\partial a_k}
    = \frac{2}{n} \sum_{i = 0}^n (Y_i - a_ke_k(x)) \cdot e_k(x) = 0 \\
    &\therefore& \, \sum_i^n Y_i e_k(x) - n a_k = 0
    \label{ans-1-before}
\end{eqnarray}

よって求める答えは式\ref{ans-1-before}より

\begin{equation}
  a^* = \frac{1}{n} \sum_i^n Y_i e_k(x)
\end{equation}

である。

\subsection{$a^*$の平均を求めよ}
平均を$E_A(\cdot)$で表す。$E_A(Y_i) = 0$を用いて、

\begin{eqnarray}
  E_A(a^*)
    &=& E_A \left(\frac{1}{n}\sum_{i=0}^n Y_i e_k (x) \right) \nonumber \\
    &=& 0
\end{eqnarray}

となる。
\subsection{}
まず、一般に$(k, l)$成分の場合の共分散を考える。$i$成分の期待値を$\mu_i$と置くと、

\begin{eqnarray}
  \sigma_{i, j}
    &=& E_A
      \left(
        \left(a^*_k - \mu_i\right)
        \left(a^*_l - \mu_l\right)
      \right) \nonumber \\
    &=& E_A\left(a^*_k \cdot a^*_l \right) \nonumber \\
    &=& E_A
      \left(
        \left(\frac{1}{n} \sum_{i = 1}^n Y_i e_k (x_i)\right)
        \left(\frac{1}{n} \sum_{i = 1}^n Y_i e_l (x_i)\right)
      \right) \nonumber \\
    &=& \frac{1}{n^2} E_A
      \left(
        \sum_{i, j = 1}^n Y_i Y_j e_k(x_i) e_l(x_j)
      \right) \label{var_ea_1}
\end{eqnarray}

となる。ここで、$E_A(Y_i) = 0, E_A(Y_i^2) = 1$であることより、

\begin{eqnarray}
  E_A\left(
    \sum_{i, j = 1}^n Y_i Y_j e_k(x_i) e_l(x_j)
  \right)
    &=& \sum_{i, j = 1}^n E_A(Y_i)E_A(Y_j) \sum_{i, j = 1}^n e_k(x_i) e_l(x_j) \nonumber \\
    &=& \begin{cases}
      \sum_{i = 1}^n E_A(Y_i^2) \sum_{i = 1}^n e_k(x_i) e_l(x_i) & (i = j) \\
      0  & (i \neq j)
    \end{cases} \nonumber \\
    &=& \begin{cases}
      n^2 & (i = j) \\
      0 & (i \neq j)
    \end{cases} \label{var_ea_2}
\end{eqnarray}

となる。したがって、求める分散共分散行列 $\Sigma$ は、(\ref{var_ea_1})と(\ref{var_ea_2})より

\begin{equation}
  \Sigma = \left(
    \begin{array}{cccc}
      1 & 0 & \ldots & 0 \\
      0 & 1 & \ldots & 0 \\
      \vdots & \vdots & \ddots & \vdots \\
      0 & 0 & \ldots & 1
    \end{array}
  \right) = I
\end{equation}

と求まる。

\subsection{}


\section{応用問題}

\subsection{}
「市町村2012estat.csv」に対し、回帰分析、主成分分析とクラスタ分析を用いて解析を行った。

\subsubsection{回帰分析}



\end{document}
